\documentclass{sysubeamer}

\pretitle{Version 1.0}
\title{Beamer Template For \\ Sun Yat-sen University}
\subtitle{Using \LaTeX\ to prepare slides}
\author{LulietLyan}

\begin{document}
\maketitle

\chapter{Introduction}

\section{Single Column}

Write your presentation like a normal \LaTeX\ file with a \verb|\maketitle|
command and \verb|\chapter| and \verb|\section| headings. The \verb|\maketitle|
contents are defined by the following macros:
\begin{center}
    \begin{tabular}{l@{\qquad}l@{\qquad}l}
        \verb|\pretitle| &
        \verb|\title| &
        \verb|\subtitle| \\
        \verb|\author| &
        \verb|\extralogo|
    \end{tabular}
\end{center}
The \verb|\extralogo| command specifies an extra logo below the SYSU-BEAMER crest. The
\verb|\chapter| heading creates a slide with just the chapter name. The
\verb|\section| heading sets the title of a new slide. However, if no text
follows the section, no slide will be created. Text which does not fit on one
slide will flow onto the next slide automatically. To get 4-by-3 aspect ratio
slides, specify \verb|standard| as an option to the document class.

\section{Double Column}\twocolumn\raggedright

Use the \verb|\twocolumn| and \verb|\onecolumn| commands right after the section
heading to control the number of columns. Text will flow from the left column to
the right.
\begin{itemize}
    \item Point one
    \item Point two
    \item Point three
    \item Point four
    \item Point five
    \item Point six
    \item Point seven
    \item Point eight
    \item Point nine
    \item Point ten
    \item Point eleven
    \item Point twelve
\end{itemize}
You can use \verb|\pagebreak| to force text onto the next column.

\section{Table of Stuff}

You can create any variety of subdivisions on your slide by using the
\verb|tabular| environment.
\begin{center}
\begin{tabular}{C{0.25\textwidth}cC{0.25\textwidth}cC{0.25\textwidth}}
    \cellcolor{dark}\textcolor{white}{Primary} &&
    \cellcolor{dark}\textcolor{white}{Secondary} &&
    \cellcolor{dark}\textcolor{white}{Tertiary} \\
    First && Second && Third \\
    One && Two && Three \\[1em]
    \cellcolor{dark}\textcolor{white}{Alpha} &&
    \cellcolor{dark}\textcolor{white}{Beta} &&
    \cellcolor{dark}\textcolor{white}{Gamma} \\
    Green && Blue && Red \\
    Cyan && Yellow && Magenta
\end{tabular}
\end{center}
The \verb|\cellcolor| command sets the background color of a table cell.

\section{Quad Charts}

\quadchart{% Top-left content
    \textbf{Itemized Lists}\scriptsize\vspace{0.5cm}

    \begin{itemize}
        \item Budget: \$1,000,000
        \item Spent to Date: \$725,000
        \item Remaining: \$275,000
        \item Burn Rate: \$150,000/mo.
        \item Projection: On track
    \end{itemize}
};
        }
    \end{tikzpicture}
    \end{center}
}{% Bottom-left content
    \textbf{Tables}\tiny\vspace{1em}

    \begin{center}
    \renewcommand{\arraystretch}{1.3}
    \begin{tabular}{@{}lrrr@{}}
        \hline
        \textbf{Category} & \textbf{Budget} &
            \textbf{Spent} & \textbf{Remaining} \\
        \hline
        Salaries          & \$500,000       & \$325,000      & \$175,000 \\
        Equipment         & \$200,000       & \$180,000      & \$20,000  \\
        Travel            & \$50,000        & \$30,000       & \$20,000  \\
        Marketing         & \$150,000       & \$125,000      & \$25,000  \\
        Miscellaneous     & \$100,000       & \$65,000       & \$35,000  \\
        \hline
        \textbf{Total}    & \textbf{\$1,000,000} &
            \textbf{\$725,000} & \textbf{\$275,000} \\
        \hline
    \end{tabular}
    \end{center}
}{% Bottom-right content
    \textbf{Text}\vspace{0.7em}

    \hspace{0.6cm}
    \parbox{0.8\textwidth}{\tiny%
    Unless absolutely required, avoid quad charts. Their four-quadrant structure often becomes overloaded with text, data, and visuals, making them visually cluttered and difficult to read. The limited space in each quadrant forces critical details to compete for attention, overwhelming audiences and obscuring key points. This density, combined with small fonts and cramped layouts, creates a readability nightmare, especially for those trying to quickly grasp the content, ultimately hindering clear communication. You are far better off using four separate slides.}
}

\section{Centering}

\begin{Center}
    \Large Use the \texttt{Center} environment \\
    to center horizontally \emph{and} vertically.
\end{Center}

\chapter{Explicit Code}

\section{Python}\onecolumn

Use the \verb|python| environment for Python code.
\begin{python}
def write_list(fid, x, level):
    ind = '  '*level
    xs = '0' if abs(x[0]) < 1e-3 else "%.3f"
`\HL`    txt = '\n%svalues=\"%s' % (ind, xs)
    for n in range(1, len(x)):
        xs = '0' if abs(x[n]) < 1e-3 else "%.3f"
        if len(txt) + 3 + len(xs) >= 80:
            fid.write(txt + ';\n')
            txt = ind + '  ' + xs
        else:
            txt += '; ' + xs
    fid.write(txt + '\"')
\end{python}

\section{MATLAB}

Use the \verb|matlab| environment for MATLAB code.
\begin{matlab}
function savepdf(name, width, height)
    % name is the file name including ".pdf".
    % Both width and height are in (cm).
    set(gcf, 'units', 'centimeters', ...
        'position', [0, 0, width, height])
    set(gca, 'FontSize', 9);
    set(gca, 'FontName', 'Times New Roman');
    exportgraphics(gcf, name, ...
        'ContentType', 'vector');
end
\end{matlab}

\section{R Language}

Use the \verb|rlang| environment for R code.
\begin{rlang}
factorial <- function(n) {
    if (n == 0 || n == 1) {
        return(1)
    } else {
        return(n * factorial(n - 1))
    }
}
\end{rlang}

\section{Pseudocode}

Use the \verb|pseudocode| environment for non-language-specific code.
\begin{pseudocode}
function add_arrays($a$, $b$, $N$)
    $c \gets zeros(N)$
    for $n$ in $0:N-1$
        if $a_n$ and $b_n$ are real
            $c_n = a_n + b_n$
        end if
    end for
    return $c$
end function
\end{pseudocode}

\chapter{Control and Classification}

\section{Controlled Unclassified Information}

Unless your presentation is being distributed, no markings need to be applied.
If it is approved for public release without restriction, you can mark it as
Distribution A with the \verb|\distributionA| command in the preamble.

If it is approved with a different distribution statement (B through F), specify
the banner (markings in the header and footer) (e.g., \verb|\banner{cui}|) and
fill in the details with the \verb|\cui| command:
\begin{verbatim}
  \cui{Controlled By: AETC \\
      Controlled By: SYSU-BEAMER/ENG \\
      CUI Category(ies): PRVCY \\
      Distribution: \DistB{CATEGORY}{DATE}{OFFICE} \\
      POC: John Smith, 555-123-4567}
\end{verbatim}

\section{Classified Information}

For classified information, use the \verb|\banner| command and the
classification (e.g., \verb|\banner{secret}|) and the \verb|\classified|
command:
\begin{verbatim}
  \classified{
      Classified By: \\
      Derived From: \\
      Declassify On: }
\end{verbatim}

\section{Color Themes}

Although specific colors are not officially dictated, it is common to use
certain colors for certain degrees of information control. A color theme can be
set for the presentation by entering the color as a parameter to the class:
\verb|\documentclass[purple]{afitdefense}|. The commonly used colors are
\begin{center}
    \begin{tabular}{l|lc@{\qquad}l|l}
        \color{yellow}Top Secret//SCI & \color{yellow}yellow
        && \color{blue}Confidential & \color{blue}blue \\
        \color{orange}Top Secret & \color{orange}orange
        && \color{purple}CUI & \color{purple}purple \\
        \color{red}Secret & \color{red}red
        && \color{green}Uncontrolled & \color{green}green
    \end{tabular}
\end{center}
Note that ``CUI'' means ``Controlled Unclassified Information.''

% You can create blank pages for placing full-page graphics or text with the
% `\blankpage` command.
\blankpage
% You can place images and text at arbitrary locations with the `\pos` command.
\pos{0pt}{0pt}{\tikz{
    \newcounter{density}\setcounter{density}{10}
    \def\mainColor{light}
    \path[coordinate] (0,0) coordinate(A)
        ++(\paperwidth,0) coordinate(B)
        ++(0,-\paperheight) coordinate(C)
        ++(-\paperwidth,0) coordinate(D);
    \fill[\mainColor!\thedensity]
        (A) -- (B) -- (C) -- (D) -- cycle;
    \foreach \x in {1,...,18}{%
        \pgfmathsetcounter{density}{\thedensity+5}
        \setcounter{density}{\thedensity}
        \path[coordinate] coordinate(X) at (A){};
        \path[coordinate] (A)
            -- (B) coordinate[pos=0.15](A)
            -- (C) coordinate[pos=0.15](B)
            -- (D) coordinate[pos=0.15](C)
            -- (X) coordinate[pos=0.15](D);
        \draw[\mainColor!80!black, fill=\mainColor!\thedensity]
            (A) -- (B) -- (C) -- (D) -- cycle;
    }
}}
\pos[6cm]{0.125\textwidth}{2cm}{
    \raggedright\large Make blank pages with the ``\texttt{\textbackslash
    blankpage}'' command.
}
\pos[8cm]{0.5\textwidth}{6cm}{
    \raggedright\large Arbitrarily position content with the ``\texttt{\textbackslash
    pos[width]\{x\}\{y\}\{content\}}'' command.
}

\closing
\end{document}
